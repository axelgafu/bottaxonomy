%%%%%%%%%%%%%%%%%%%%%%%%%%%%%%%%%%%%%%%%%
% Journal Article
% LaTeX Template
% Version 1.3 (9/9/13)
%
% This template has been downloaded from:
% http://www.LaTeXTemplates.com
%
% Original author:
% Frits Wenneker (http://www.howtotex.com)
%
% License:
% CC BY-NC-SA 3.0 (http://creativecommons.org/licenses/by-nc-sa/3.0/)
%
%%%%%%%%%%%%%%%%%%%%%%%%%%%%%%%%%%%%%%%%%

%----------------------------------------------------------------------------------------
%	PACKAGES AND OTHER DOCUMENT CONFIGURATIONS
%----------------------------------------------------------------------------------------

\documentclass[twoside]{article}

\usepackage{lipsum} % Package to generate dummy text throughout this template

\usepackage[sc]{mathpazo} % Use the Palatino font
\usepackage[T1]{fontenc} % Use 8-bit encoding that has 256 glyphs
\linespread{1.05} % Line spacing - Palatino needs more space between lines
\usepackage{microtype} % Slightly tweak font spacing for aesthetics

\usepackage[hmarginratio=1:1,top=32mm,columnsep=20pt]{geometry} % Document margins
\usepackage{multicol} % Used for the two-column layout of the document
\usepackage[hang, small,labelfont=bf,up,textfont=it,up]{caption} % Custom captions under/above floats in tables or figures
\usepackage{booktabs} % Horizontal rules in tables
\usepackage{float} % Required for tables and figures in the multi-column environment - they need to be placed in specific locations with the [H] (e.g. \begin{table}[H])
\usepackage{hyperref} % For hyperlinks in the PDF

\usepackage{lettrine} % The lettrine is the first enlarged letter at the beginning of the text
\usepackage{paralist} % Used for the compactitem environment which makes bullet points with less space between them

\usepackage{abstract} % Allows abstract customization
\renewcommand{\abstractnamefont}{\normalfont\bfseries} % Set the "Abstract" text to bold
\renewcommand{\abstracttextfont}{\normalfont\small\itshape} % Set the abstract itself to small italic text

\usepackage{titlesec} % Allows customization of titles
\renewcommand\thesection{\Roman{section}} % Roman numerals for the sections
\renewcommand\thesubsection{\Roman{subsection}} % Roman numerals for subsections
\titleformat{\section}[block]{\large\scshape\centering}{\thesection.}{1em}{} % Change the look of the section titles
\titleformat{\subsection}[block]{\large}{\thesubsection.}{1em}{} % Change the look of the section titles

\usepackage{fancyhdr} % Headers and footers
\pagestyle{fancy} % All pages have headers and footers
\fancyhead{} % Blank out the default header
\fancyfoot{} % Blank out the default footer
\fancyhead[C]{May 2015 $\bullet$ UAG} % Custom header text
%\fancyfoot[RO,LE]{\thepage} % Custom footer text

\usepackage{relsize}
\usepackage{hyperref}

%----------------------------------------------------------------------------------------
%	TITLE SECTION
%----------------------------------------------------------------------------------------

\title{\vspace{-15mm}\fontsize{24pt}{10pt}\selectfont\textbf{Bot Taxonomy Proposal}} % Article title

\author{
\large
\textsc{Graded Student in Computer Science Axel Alejandro Garcia Fuentes}\thanks{-}\\[2mm] % Your name
\normalsize Universidad Autonoma de Guadalajara \\ % Your institution
\normalsize \href{mailto:axel.garcia@edu.uag.mx}{axel.garcia@edu.uag.mx} % Your email address
\vspace{-5mm}
}
\date{}

%----------------------------------------------------------------------------------------

\begin{document}

\maketitle % Insert title

\thispagestyle{fancy} % All pages have headers and footers

%----------------------------------------------------------------------------------------
%	ABSTRACT
%----------------------------------------------------------------------------------------

\begin{abstract}

%\noindent \lipsum[1] % Dummy abstract text
Bots are used to automate several things. The applicability of bots range from perform editing activities to mimic the human behavior.
This document has a bot taxonomy proposal based on their characteristics. This article was motivated by the idea that a bot taxonomy 
may help to study their capabilities and properties.
\end{abstract}

%----------------------------------------------------------------------------------------
%	ARTICLE CONTENTS
%----------------------------------------------------------------------------------------

\begin{multicols}{2} % Two-column layout throughout the main article text

\section{Introduction}
\lettrine[nindent=0em,lines=3]{C}omputers help humans to speed up computations and to automate things. One of the ways in which
computers help us to automate tasks is by creating a program that does it. Software bots are computer programs that perform a given 
task they were programmed for. 

There are bots that are able to write articles. In this category falls a bot called Lsjbot. That boot created about 454,000 articles and that
is bout the half of the articles in Swedish Wikipedia \cite{guld:2013}. Another example is ClueBot NG, which is a bot that cleans up
vandalism from articles. \cite{bbc07:2012} mentions a situation where that bot identified a message that belonged nowhere in an article
of the national supreme court.

Some of them are so sophisticated that can even answer queries by using natural language algorithms\cite{emerging:2014}. It seems
there are two different types of bots: Social bots and editing bots. When multiple social bots are controlled by a person that is called
Sybil\cite{ferrara:2015}.
%\cite{kuhn:2015} Science Bots: a Model for the Future of Scientific Computation?: http://arxiv.org/pdf/1503.04374.pdf

\cite{kuhn:2015} mentions botnets which seems to be similar to Sybil. The difference seems to be that botnets are intended to be
used with malware and Sybils are social bots.

\cite{ferrara:2015} mentions that early bots mainly modified content automatically, examples of that are \cite{wikiWriter:2014} and \cite{wikiList:2014}. It is also mentioned the existence of social bots which is considered as a computer algorithm that automatically produces content and interacts with humans on social media; e.g. Twitter or Facebok.
Two main categories of bots: Social Bots and Work Bots. A social bot may work along with other social bots.

According with \cite{ferrara:2015} exist bots that are aimed to mimic humans. While there are benign bots they can be created to persuade, smear or deceive also. 
Real users seem to spend more time looking at other user's contents and messaging than Sybil.

Wikipedia page about creating wikipedia bots to aid article creation\cite{wtalk:2014} and this other to request the creation of a bot: \url{http://en.wikipedia.org/wiki/Wikipedia:Bot_requests}

---
How good boots are used: Wikipedia Bot Writes 10,000 Articles a Day: \url{http://news.discovery.com/tech/robotics/wikipedia-bot-writes-10000-articles-a-day-140715.htm}

%------------------------------------------------
\section{Problem}
Mention section Engineered social tampering of The Rise of Social Bots: \url{http://arxiv.org/pdf/1407.5225v2.pdf}
%Reference \cite{saif.ea:2012} mentions 
\cite{emerging:2014} believes bots may be less easy to detect as the time goes on. That reference mentions cases where bots and humans work together and the bot uses the account of a human to publish messages with Twitter; there may be cases where the human account is hacked and the bot publishes messages from a hacked account.

%------------------------------------------------
\section{Taxonomies}
Some of them are benign and, in principle, innocuous
or even helpful: this category includes bots that automatically aggregate content
from various sources, like simple news feeds\cite{ferrara:2015}
    
Editing bots:
    According with \cite{kuhn:2015} bots can be independent pieces of software which are capable of perform small tasks like create nanoposts. 
    Swedish Wikipedia surpasses 1 million articles with aid of article creation bot: http://blog.wikimedia.org/2013/06/17/swedish-wikipedia-1-million-articles/

Social bots:
    Distractors[Abokhodair et al. 2015].Smoke screening strategies. Political campaigns orchestrated by social bots[Ratkiewicz et al. 2011a].
    Bots-Human (maybe cyborg?) bot
    Bots that hack accounts vs
    A second category of social bots includes malicious entities designed specifically with the purpose to harm\cite{ferrara:2015}; inflate support for 
    a political candidate. In fact, these kinds of abuse have already been observed: during the 2010 U.S. midterm elections [Ratkiewicz et al. 2011a];. 
    Campaigns of this type are sometimes referred to as astroturf or Twitter bombs\cite{ferrara:2015}.
    
Botnets:
    \cite{abu:2006} refers to botnets as a network of infected hosts which are called \emph{bots}. In this context those bots are controlled by a human
    operator which is called \emph{botmaster}. That definition is similar to the definition of \emph{Sybil} \cite{ferrara:2015}. However, a botnet
    is meant to be utilized by malware according to\cite{abu:2006}.



%Marc Gaffan, Top 10 Bots You Should Know About, Aug/21/2012, last access May/05/2015, https://www.incapsula.com/blog/know-your-top-10-bots.html
Search Bot, Crawler, Feed Fetcher.

%Discuss taxonomies:
% - Wikipedia:Types of bots: http://en.wikipedia.org/wiki/Wikipedia:Types_of_bots
% - Different Types of Bots: http://www.honeynet.org/node/53
% - Different Types of Internet Bots and How They Are Used: http://www.spamlaws.com/how-internet-bots-are-used.html
% - Official List of Bot Types: http://realsteel.wikia.com/wiki/Official_List_of_Bot_Types
% - Bot types: http://botology.tumblr.com/types

%------------------------------------------------
\section{Detection}

innocent-by-association strategy), ?Sybil until proven otherwise? approach(opposite to the first)\cite{ferrara:2015}

%Mention Researchers Release Twitter Bot Detection Tool: http://www.tripwire.com/state-of-security/latest-security-news/researchers-release-twitter-bot-detection-tool/
%Include example of: Bot or Not? A Truthy project: http://truthy.indiana.edu/botornot/


%----------------------------------------------------------------------------------------
%	REFERENCE LIST
%----------------------------------------------------------------------------------------
%http://arxiv.org/pdf/1503.04374.pdf
%http://arxiv.org/find/all/1/all:+Bot/0/1/0/all/0/1
%http://arxiv.org/pdf/1407.5225v2.pdf
%http://truthy.indiana.edu/botornot/
%http://www.tripwire.com/state-of-security/latest-security-news/researchers-release-twitter-bot-detection-tool/
%https://www.shadowserver.org/wiki/pmwiki.php/Information/Honeypots

%https://www.google.com.mx/?gfe_rd=cr&ei=AZcLVYu7JojUpAOEmYKADA&gws_rd=ssl#q=bot+detection

The rise of social bots: Fighting deception and misinformation on social media: INDIANA UNIVERSITY: http://itnews.iu.edu/events/the-rise-of-social-bots-fighting-deception-and-misinformation-on-social-media.php

%BotoPedia, http://www.botopedia.org

% Steps for typesetting document with Bibtex:
% 1) Latex
% 2) Bibtex
% 3) Latex
% 4) Latex

\bibliographystyle{apalike} % acm, ieeetr, apalike, ...
\bibliography{biblio}

%----------------------------------------------------------------------------------------

\end{multicols}

\end{document}
